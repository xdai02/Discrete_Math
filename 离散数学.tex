\documentclass[12pt, openany, oneside]{book}

\usepackage{listings}
\usepackage[dvipsnames]{xcolor}
\usepackage{ctex}
\usepackage{fontspec}
\usepackage{setspace}
\usepackage{tikz}
\usepackage{anyfontsize}
\usepackage{sectsty}
\usepackage{titlesec}
\usepackage{float}
\usepackage[hidelinks]{hyperref}
\usepackage[a4paper]{geometry}
\usepackage{url}
\usepackage{amssymb}
\usepackage{fontawesome5}
\usepackage[most]{tcolorbox}
% \usepackage{minted}

\definecolor{mycolor}{RGB}{0,128,128}
\newtcbox{\mybox} {
    on line,
    colback=mycolor,
    fontupper=\bfseries\color{white},
    boxrule=0pt,
    arc=5pt, 
    boxsep=0pt, 
    left=2pt, 
    right=2pt, 
    top=5pt, 
    bottom=5pt
}

\usetikzlibrary{calc}
\usetikzlibrary{calc,shapes.multipart,chains,arrows}

\setstretch{1.5}
\setlength{\parindent}{0cm}

\geometry{a4paper,top=2.5cm,bottom=2.5cm}

\titleformat{\chapter}{\Huge\Huge\bfseries}{\chaptertitlename\ \thechapter{\ }}{0pt}{\Huge}{}
\titlespacing{\chapter}{0pt}{0pt}{12pt}

\definecolor{dkgreen}{rgb}{0,0.4,0}
\definecolor{gray}{rgb}{0.5,0.5,0.5}
\definecolor{mauve}{rgb}{0.58,0,0.82}
\definecolor{LightGray}{gray}{0.9}

\lstset{
    basicstyle=\normalsize \fontspec{Consolas},    %  the size of the fonts that are used for the code
    numbers=left,            % where to put the line-numbers
    numberstyle=\color{black},  % the style that is used for the line-numbers
    numbersep=10pt,                  % how far the line-numbers are from the code
    backgroundcolor=\color{white},
    showspaces=false,
    showstringspaces=false,
    showtabs=false,
    frame=single,                   % adds a frame around the code
    rulecolor=\color{black},        % if not set, the frame-color may be changed on line-breaks within not-black text (e.g. commens (green here))
    tabsize=4,                      % sets default tabsize to 2 spaces
    captionpos=t,                   % sets the caption-position to bottom
    breaklines=false,                % sets automatic line breaking
    breakatwhitespace=true,        % sets if automatic breaks should only happen at whitespace
    title=\lstname,                   % show the filename of files included with \lstinputlisting;
    % also try caption instead of title
    numberstyle=\color{black},		% line number color
    keywordstyle=\color{blue},          % keyword style
    commentstyle=\color{dkgreen},       % comment style
    stringstyle=\color{mauve},         % string literal style
    escapeinside={\%*}{*)},            % if you want to add LaTeX within your code
    morekeywords={*,...}               % if you want to add more keywords to the set
}

\begin{document}

\pagestyle{plain}

\begin{tikzpicture}[overlay,remember picture]
    % Background color
    \fill[
        black!2]
    (current page.south west) rectangle (current page.north east);

    % Rectangles
    \shade[
        left color=Dandelion,
        right color=Dandelion!40,
        transform canvas ={rotate around ={45:($(current page.north west)+(0,-6)$)}}]
    ($(current page.north west)+(0,-6)$) rectangle ++(9,1.5);

    \shade[
        left color=lightgray,
        right color=lightgray!50,
        rounded corners=0.75cm,
        transform canvas ={rotate around ={45:($(current page.north west)+(.5,-10)$)}}]
    ($(current page.north west)+(0.5,-10)$) rectangle ++(15,1.5);

    \shade[
        left color=lightgray,
        rounded corners=0.3cm,
        transform canvas ={rotate around ={45:($(current page.north west)+(.5,-10)$)}}] ($(current page.north west)+(1.5,-9.55)$) rectangle ++(7,.6);

    \shade[
        left color=orange!80,
        right color=orange!60,
        rounded corners=0.4cm,
        transform canvas ={rotate around ={45:($(current page.north)+(-1.5,-3)$)}}]
    ($(current page.north)+(-1.5,-3)$) rectangle ++(9,0.8);

    \shade[
        left color=red!80,
        right color=red!80,
        rounded corners=0.9cm,
        transform canvas ={rotate around ={45:($(current page.north)+(-3,-8)$)}}] ($(current page.north)+(-3,-8)$) rectangle ++(15,1.8);

    \shade[
        left color=orange,
        right color=Dandelion,
        rounded corners=0.9cm,
        transform canvas ={rotate around ={45:($(current page.north west)+(4,-15.5)$)}}]
    ($(current page.north west)+(4,-15.5)$) rectangle ++(30,1.8);

    \shade[
        left color=RoyalBlue,
        right color=Emerald,
        rounded corners=0.75cm,
        transform canvas ={rotate around ={45:($(current page.north west)+(13,-10)$)}}]
    ($(current page.north west)+(13,-10)$) rectangle ++(15,1.5);

    \shade[
        left color=lightgray,
        rounded corners=0.3cm,
        transform canvas ={rotate around ={45:($(current page.north west)+(18,-8)$)}}]
    ($(current page.north west)+(18,-8)$) rectangle ++(15,0.6);

    \shade[
        left color=lightgray,
        rounded corners=0.4cm,
        transform canvas ={rotate around ={45:($(current page.north west)+(19,-5.65)$)}}]
    ($(current page.north west)+(19,-5.65)$) rectangle ++(15,0.8);

    \shade[
        left color=OrangeRed,
        right color=red!80,
        rounded corners=0.6cm,
        transform canvas ={rotate around ={45:($(current page.north west)+(20,-9)$)}}]
    ($(current page.north west)+(20,-9)$) rectangle ++(14,1.2);

    % Year
    % \draw[ultra thick,gray]
    % ($(current page.center)+(5,2)$) -- ++(0,-3cm)
    node[
            midway,
            left=0.25cm,
            text width=5cm,
            align=right,
            black!75
        ]
        {
            % {\fontsize{25}{30} \selectfont \bf ANNUAL \\[10pt] REPORT}
        }
    node[
            midway,
            right=0.25cm,
            text width=6cm,
            align=left,
            orange]
        {
            % {\fontsize{72}{86.4} \selectfont 2020}
        };

    % Title
    \node[align=center] at ($(current page.center)+(0,-5)$)
    {
    {\fontsize{72}{72} \selectfont {{离散数学}}} \\[1cm]
    {\fontsize{42}{42} \selectfont {{Discrete Mathematics}}} \\[2cm]
    {\fontsize{20}{19.2} \selectfont \textcolor{orange}{ \bf 极夜酱}} \\[4pt]
    };
\end{tikzpicture}

\newpage

\setcounter{tocdepth}{1}
\tableofcontents
\thispagestyle{empty}

\newpage

\setcounter{page}{1}

\chapter{逻辑}

\section{命题}

\subsection{命题(Proposition)}

逻辑(logic)规则给出数学语句的准确含义,这些规则用来区分有效和无效的数学论证。逻辑不仅对理解数学推理十分重要,而且在计算机科学中有许多应用,逻辑可用于电路设计、程序构造、程序正确性证明等方面。 \\

命题是逻辑的基本成分,一个命题是一个具有真值(truth value)的语句,命题可以为真也可以为假,但不能既为真又为假。 \\

\begin{table}[H]
    \centering
    \setlength{\tabcolsep}{5mm}{
        \begin{tabular}{|c|c|}
            \hline
            \textbf{命题}        & \textbf{非命题}    \\
            \hline
            I have a dog.        & What day is today? \\
            \hline
            1 + 2 = 3            & Shut the door!     \\
            \hline
            Today is Wednesday.  & 1 + 2              \\
            \hline
            It is snowing today. & x + 1 = 2          \\
            \hline
        \end{tabular}
    }
\end{table}

命题习惯上用字母$ p $, $ q $, $ r $, $ s $等来表示,如果一个命题是真命题,它的真值为真,用T表示;如果一个命题是假命题,它的真值为假,用F表示。 \\

\subsection{非运算符(NOT, Negation Operator)}

非运算符$ \neg $只作用于一个命题,其作用是反转命题的真值。 \\

真值表(truth table)可以给出命题真值之间的关系,在确定由简单命题组成的命题的真值时,真值表特别有用。

\begin{table}[H]
    \centering
    \setlength{\tabcolsep}{5mm}{
        \begin{tabular}{|c|c|}
            \hline
            \textbf{$ p $} & \textbf{$ \neg p $} \\
            \hline
            T              & F                   \\
            \hline
            F              & T                   \\
            \hline
        \end{tabular}
    }
    \caption{NOT真值表}
\end{table}

\begin{tcolorbox}
    \mybox{Exercise}
    $ \neg p $ \\
    $ p $: It snowed last night. \\
    $ \neg p $: It didn;t snow last night. \\
    $ q $: $ 2 + 3 = 6 $ \\
    $ \neg q $: $ 2 + 3 \ne 6 $
\end{tcolorbox}

\subsection{合取运算符(AND, Conjunction Operator)}

命题$ p \wedge q $表示$ p $并且$ q $,当$ p $和$ q $都为真时命题为真,否则为假。

\begin{table}[H]
    \centering
    \setlength{\tabcolsep}{5mm}{
        \begin{tabular}{|c|c|c|}
            \hline
            \textbf{$ p $} & \textbf{$ \neg p $} & \textbf{$ p \wedge q $} \\
            \hline
            T              & T                   & T                       \\
            \hline
            T              & F                   & F                       \\
            \hline
            F              & T                   & F                       \\
            \hline
            F              & F                   & F                       \\
            \hline
        \end{tabular}
    }
    \caption{AND真值表}
\end{table}

\begin{tcolorbox}
    \mybox{Exercise}
    $ p \wedge q $ \\
    $ p $: 今天是星期五。 \\
    $ q $: 今天会下雨。 \\
    $ p \wedge q $: 今天是星期五并且会下雨。
\end{tcolorbox}

\subsection{析取运算符(OR, Disjunction Operator)}

命题$ p \vee q $表示$ p $或$ q $,当$ p $和$ q $都为假时命题为假,否则为真。

\begin{table}[H]
    \centering
    \setlength{\tabcolsep}{5mm}{
        \begin{tabular}{|c|c|c|}
            \hline
            \textbf{$ p $} & \textbf{$ \neg p $} & \textbf{$ p \vee q $} \\
            \hline
            T              & T                   & T                     \\
            \hline
            T              & F                   & T                     \\
            \hline
            F              & T                   & T                     \\
            \hline
            F              & F                   & F                     \\
            \hline
        \end{tabular}
    }
    \caption{OR真值表}
\end{table}

\begin{tcolorbox}
    \mybox{Exercise}
    $ p \vee q $ \\
    $ p $: 开关坏了。 \\
    $ q $: 灯泡坏了。 \\
    $ p \vee q $: 开关坏了或者灯泡坏了。
\end{tcolorbox}

\subsection{异或运算符(XOR, Exclusive Or)}

命题$ p \oplus q $表示$ p $和$ q $的异或,当$ p $和$ q $中恰有一个为真时命题为真,否则为假。

\begin{table}[H]
    \centering
    \setlength{\tabcolsep}{5mm}{
        \begin{tabular}{|c|c|c|}
            \hline
            \textbf{$ p $} & \textbf{$ \neg p $} & \textbf{$ p \oplus q $} \\
            \hline
            T              & T                   & F                       \\
            \hline
            T              & F                   & T                       \\
            \hline
            F              & T                   & T                       \\
            \hline
            F              & F                   & F                       \\
            \hline
        \end{tabular}
    }
    \caption{XOR真值表}
\end{table}

\begin{tcolorbox}
    \mybox{Exercise}
    $ p \oplus q $ \\
    $ p $: 他现在在上海。 \\
    $ q $: 他现在在北京。 \\
    $ p \vee q $: 他现在在上海或北京。
\end{tcolorbox}

\begin{tcolorbox}
    \mybox{Exercise}
    某地发生了一件谋杀案,警察通过排查确定杀人凶手必为4个嫌疑犯的一个,根据以下信息确定凶手。 \\
    A说:不是我。 \\
    B说:是C。 \\
    C说:是D。 \\
    D说:C在胡说。 \\
    已知3个人说了真话,1个人说的是假话。
\end{tcolorbox}

\vspace{-0.5cm}
\begin{lstlisting}[language=Python]
def main():
    for killer in ['A', 'B', 'C', 'D']:
        if (killer != 'A') + (killer == 'C') \
            + (killer == 'D') + (killer != 'D') == 3:
            print(killer)

if __name__ == "__main__":
    main()
\end{lstlisting}

\begin{tcolorbox}
    \mybox{运行结果}
    C
\end{tcolorbox}

\newpage

\section{复合命题}

\subsection{复合命题(Compound Proposition)}

使用非运算符和已定义的各联结词可以构造复合命题。小括号用于规定复合命题中多个逻辑运算符的操作顺序,为了减少所需的小括号数量,规定了各联结词的优先级。

\begin{table}[H]
    \centering
    \setlength{\tabcolsep}{5mm}{
        \begin{tabular}{|c|c|}
            \hline
            \textbf{优先级} & \textbf{运算符}                   \\
            \hline
            1               & $ \neg $                          \\
            \hline
            2               & $ \wedge $ / $ \vee $             \\
            \hline
            3               & $ \rightarrow / \leftrightarrow $ \\
            \hline
        \end{tabular}
    }
    \caption{运算符优先级}
\end{table}

\subsection{蕴含运算符(Implication Operator)}

命题$ p \rightarrow q $表示$ p $蕴含$ q $,在$ p $为真而$ q $为假时命题为假,否则为真。其中$ p $称为前提,$ q $称为结论。

\begin{table}[H]
    \centering
    \setlength{\tabcolsep}{5mm}{
        \begin{tabular}{|c|c|c|}
            \hline
            \textbf{$ p $} & \textbf{$ q $} & \textbf{$ p \rightarrow q $} \\
            \hline
            T              & T              & T                            \\
            \hline
            T              & F              & F                            \\
            \hline
            F              & T              & T                            \\
            \hline
            F              & F              & T                            \\
            \hline
        \end{tabular}
    }
    \caption{蕴含真值表}
\end{table}

表示$ p \rightarrow q $的术语有很多种,常见的有:

\begin{itemize}
    \item If $ p $, then $ q $.
    \item $ p $ only if $ q $.
    \item $ q $ is necessary for $ p $.
\end{itemize}

\begin{tcolorbox}
    \mybox{Exercise}
    $ p \rightarrow q $ \\
    $ p $:我去看电影。 \\
    $ q $:我买奶茶。 \\
    $ p \rightarrow q $:如果我去看电影,那么我会买奶茶。
\end{tcolorbox}

\begin{figure}[H]
    \centering
    \includegraphics[scale=0.7]{img/C1/1-2/1.png}
\end{figure}

由$ p \rightarrow q $可以构造出几个相关的蕴含:

\begin{enumerate}
    \item $ q \rightarrow p $称为$ p \rightarrow q $的逆命题(converse)。
    \item $ \neg q \rightarrow \neg p $称为$ p \rightarrow q $的逆否命题(contrapositive)。
\end{enumerate}

\begin{tcolorbox}
    \mybox{Exercise}
    逆命题与逆否命题 \\
    $ p $:今天是星期四。 \\
    $ q $:我今天有考试。 \\
    $ p \rightarrow q $:如果今天是星期四,那么我今天有考试。 \\
    $ q \rightarrow p $:如果我今天有考试,那么果今天是星期四。 \\
    $ \neg q \rightarrow \neg p $:如果我今天没有考试,那么今天不是星期四。
\end{tcolorbox}

\subsection{双向蕴含(Biconditional Operation)}

命题$ p \leftrightarrow q $表示$ p $双向蕴含$ q $,在$ p $和$ q $有相同的真值时命题为真,否则为假。

\begin{table}[H]
    \centering
    \setlength{\tabcolsep}{5mm}{
        \begin{tabular}{|c|c|c|}
            \hline
            \textbf{$ p $} & \textbf{$ q $} & \textbf{$ p \leftrightarrow q $} \\
            \hline
            T              & T              & T                                \\
            \hline
            T              & F              & F                                \\
            \hline
            F              & T              & F                                \\
            \hline
            F              & F              & T                                \\
            \hline
        \end{tabular}
    }
    \caption{双向蕴含真值表}
\end{table}

双蕴含的真值表与异或的真值表正好相反,因此$ p \leftrightarrow q $与$ \neg (p \oplus q) $等价。

\newpage

\section{逻辑等价}

\subsection{逻辑等价(Logical Equivalence)}

两个不同的复合命题可能拥有完全相同的真值,则称这两个命题在逻辑上是等价的。如果无论复合命题中各个命题的真值是什么,复合命题的真值总是为真,这样的复合命题称为永真式(tautology)。如果真值永远为假的复合命题称为矛盾(contradiction)。

\begin{table}[H]
    \centering
    \setlength{\tabcolsep}{5mm}{
        \begin{tabular}{|c|c|c|c|}
            \hline
            \textbf{$ p $} & \textbf{$ \neg p $} & \textbf{$ p \vee \neg p $} & \textbf{$ p \wedge \neg p $} \\
            \hline
            T              & F                   & T                          & F                            \\
            \hline
            F              & T                   & T                          & F                            \\
            \hline
        \end{tabular}
    }
    \caption{逻辑等价}
\end{table}

如果复合命题$ s $和是$ r $逻辑等价的,可表示为$ s \equiv r $。只有当$ s \leftrightarrow r $是永真式时,$ s $和$ r $才是逻辑等价的。

\begin{tcolorbox}
    \mybox{Exercise}
    使用真值表证明$ p \vee q \equiv \neg (\neg p \wedge \neg q) $
    \begin{table}[H]
        \centering
        \setlength{\tabcolsep}{5mm}{
            \begin{tabular}{|c|c|c|c|c|c|c|}
                \hline
                \textbf{$ p $} & \textbf{$ q $} & \textbf{$ p \vee q $} & \textbf{$ \neg p $} & \textbf{$ \neg q $} & \textbf{$ \neg p \wedge \neg q $} & \textbf{$ \neg (\neg p \wedge \neg q) $} \\
                \hline
                T              & T              & T                     & F                   & F                   & F                                 & T                                        \\
                \hline
                T              & F              & T                     & F                   & T                   & F                                 & T                                        \\
                \hline
                F              & T              & T                     & T                   & F                   & F                                 & T                                        \\
                \hline
                F              & F              & F                     & T                   & T                   & T                                 & F                                        \\
                \hline
            \end{tabular}
        }
    \end{table}
\end{tcolorbox}

\subsection{逻辑等价定理}

\begin{tcolorbox}
    \mybox{幂等律 Idempotent Laws}
    \begin{align}
        p \wedge p & \equiv p \\
        p \vee p   & \equiv p
    \end{align}
\end{tcolorbox}

\begin{tcolorbox}
    \mybox{恒等律 Identity Laws}
    \begin{align}
        p \wedge T & \equiv p \\
        p \vee F   & \equiv p
    \end{align}
\end{tcolorbox}

\begin{tcolorbox}
    \mybox{支配律 Domination Laws}
    \begin{align}
        p \vee T   & \equiv T \\
        p \wedge F & \equiv F
    \end{align}
\end{tcolorbox}

\begin{tcolorbox}
    \mybox{双非律 Double Negation Law}
    \begin{align}
        \neg (\neg p) & \equiv p
    \end{align}
\end{tcolorbox}

\begin{tcolorbox}
    \mybox{交换律 Commutative Laws}
    \begin{align}
        p \wedge q & \equiv q \wedge p \\
        p \vee q   & \equiv q \vee p
    \end{align}
\end{tcolorbox}

\begin{tcolorbox}
    \mybox{结合律 Associative Laws}
    \begin{align}
        (p \wedge q) \wedge r & \equiv p \wedge (q \wedge r) \\
        (p \vee q) \vee r     & \equiv p \vee (q \vee r)
    \end{align}
\end{tcolorbox}

\begin{tcolorbox}
    \mybox{分配率 Distributive Laws}
    \begin{align}
        (p \wedge q) \wedge r & \equiv p \wedge (q \wedge r) \\
        (p \vee q) \vee r     & \equiv p \vee (q \vee r)
    \end{align}
\end{tcolorbox}

\begin{tcolorbox}
    \mybox{德摩根律 De Morgan's Laws}
    \begin{align}
        \neg (p \wedge q) & \equiv \neg p \vee \neg q   \\
        \neg (p \vee q)   & \equiv \neg p \wedge \neg q
    \end{align}
\end{tcolorbox}

\begin{tcolorbox}
    \mybox{吸收律 Absorption Laws}
    \begin{align}
        p \wedge (p \vee q) & \equiv p \\
        p \vee (p \wedge q) & \equiv p
    \end{align}
\end{tcolorbox}

\begin{tcolorbox}
    \mybox{条件恒等}
    \begin{align}
        p \rightarrow q     & \equiv \neg p \vee q                              \\
        p \leftrightarrow q & \equiv (p \rightarrow q) \wedge (q \rightarrow p)
    \end{align}
\end{tcolorbox}

\begin{tcolorbox}\nonumber
    \mybox{Exercise}
    证明$ (p \vee q) \rightarrow p $永真
    \begin{align}
         & (p \vee q) \rightarrow p           \\
         & \equiv \neg (p \wedge q) \vee p    \\
         & \equiv (\neg p \vee \neg q) \vee p \\
         & \equiv (\neg q \vee \neg p) \vee p \\
         & \equiv \neg q \vee (\neg p \vee p) \\
         & \equiv \neg q \vee T               \\
         & \equiv T
    \end{align}
\end{tcolorbox}

\newpage

\section{谓词与量词}

\subsection{谓词(Predicate)}

命题逻辑并不能表达数学语言和自然语言中所有语句的确切含义。在数学表达式和计算机程序中经常可以看到含有变量的语句,例如$ x > 3 $、$ x = y + 3 $、程序$ x $正在运行等。当变量值未指定时,这些语句既不为真也不为假。 \\

利用$ P(x) $可以表示语句,其中$ x $是变量,语句$ P(x) $可以说是命题函数$ P $在$ x $的值。一旦给变量$ x $赋一个值,语句$ P(x) $就称为命题并具有真值。 \\

通常使用大写字母$ P $, $ Q $, $ R $等表示谓词,小写字母$ x $, $ y $, $ z $等表示变量。

\begin{tcolorbox}
    \mybox{Exercise}
    谓词
    \begin{table}[H]
        \centering
        \setlength{\tabcolsep}{5mm}{
            \begin{tabular}{|l|l|}
                \hline
                \textbf{谓词}          & \textbf{真值}      \\
                \hline
                $ P(x): x + 3 = 6 $    & $ P(3) $为True     \\
                \hline
                $ Q(x, y): x = y + 2 $ & $ Q(4, 1) $为False \\
                \hline
            \end{tabular}
        }
    \end{table}
\end{tcolorbox}

\subsection{量词(Quantifier)}

量词用量化表示在何种程度上谓词对于一定范围的个体成立。量词有全称量词(universal quantifier)和存在量词(existential quantifer)。 \\

全称量词$ \forall $表示all。$ \forall_x P(x) $是一个命题,当范围内所有的$ x $都能使语句$ P(x) $为真时,命题为真。

$$
    \forall_x P(x) = P(a_1) \wedge P(a_2) \wedge \dots \wedge P(a_k)
$$

\begin{tcolorbox}
    \mybox{Exercise}
    全称量词 \\
    假设$ x $表示全班所有学生,$ P(x) $表示$ x $完成了作业。 \\
    $ \forall_x P(x) $:全班所有学生都完成了作业。
\end{tcolorbox}

存在量词$ \exists $表示exists。$ \exists_x P(x) $是一个命题,当范围内存在至少一个$ x $能够语句$ P(x) $为真时,命题为真。

$$
    \exists_x P(x) = P(a_1) \vee P(a_2) \vee \dots \vee P(a_k)
$$

\begin{tcolorbox}
    \mybox{Exercise}
    存在量词 \\
    假设$ x $表示全班所有学生,$ P(x) $表示$ x $完成了作业。 \\
    $ \exists_x P(x) $:班里存在有一个学生完成了作业。
\end{tcolorbox}

\begin{tcolorbox}
    \mybox{Exercise}
    嵌套量词 \\
    假设$ x $表示某个人,$ P(x) $表示有父母。 \\
    $ \forall_x P(x) $:所有人都有父母。 \\
    $ \exists_x \neg P(x) $:存在至少有一个人没有父母。 \\
    $ \exists_x \exists_y (P(x) \wedge P(y)) $:至少存在一个人$ x $和一个人$ y $有父母。
\end{tcolorbox}

\begin{tcolorbox}
    \mybox{Exercise}
    $ P(x) $:$ x $是偶数,$ Q(x) $:$ x $能被3整除,$ x \in \mathbb{Z}^+ $
    \begin{table}[H]
        \centering
        \setlength{\tabcolsep}{5mm}{
            \begin{tabular}{|l|l|}
                \hline
                \textbf{语句}                              & \textbf{真值} \\
                \hline
                $ \exists_x (P(x) \wedge Q(x)) $           & True          \\
                \hline
                $ \forall_x (P(x) \rightarrow \neg Q(x)) $ & False         \\
                \hline
            \end{tabular}
        }
    \end{table}
\end{tcolorbox}

\subsection{全称量词的否定}

否定运算符可以使用在全称量词上。

\begin{align}
    \neg \forall_x P(x) & \equiv \exists_x \neg P(x) \\
    \neg \exists_x P(x) & \equiv \forall_x \neg P(x)
\end{align}

\begin{tcolorbox}
    \mybox{Exercise}
    全称量词的否定 \\
    $ P(x) $: $ x $ will pass the course ($ x $ is a student). \\
    $ \neg \forall_x P(x) $: Not all students will pass the course. \\
    $ \forall_x \neg P(x) $: No student will pass the course. \\
    $ \neg \exists_x P(x) $: There does not exist a student that will pass the course. \\
    $ \exists_x \neg P(x) $: There exists a student that will not pass the course.
\end{tcolorbox}

\newpage

\section{证明}

\subsection{证明(Proof)}

证明方法非常重要,不仅因为它们可用于证明数学定理,而且在计算机科学中也有许多应用,包括验证程序正确性、建立安全的操作系统、人工智能领域做推论等。证明就是建立定理真实性的有效论证。 \\

证明定理有很多方法:



\end{document}