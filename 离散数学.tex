\documentclass[12pt, openany, oneside]{book}

\usepackage{listings}
\usepackage[dvipsnames]{xcolor}
\usepackage{ctex}
\usepackage{fontspec}
\usepackage{setspace}
\usepackage{tikz}
\usepackage{anyfontsize}
\usepackage{sectsty}
\usepackage{titlesec}
\usepackage{float}
\usepackage[hidelinks]{hyperref}
\usepackage[a4paper]{geometry}
\usepackage{url}
\usepackage{amssymb}
\usepackage{fontawesome5}
\usepackage[most]{tcolorbox}
% \usepackage{minted}

\definecolor{mycolor}{RGB}{0,128,128}
\newtcbox{\mybox} {
    on line,
    colback=mycolor,
    fontupper=\bfseries\color{white},
    boxrule=0pt,
    arc=5pt, 
    boxsep=0pt, 
    left=2pt, 
    right=2pt, 
    top=5pt, 
    bottom=5pt
}

\usetikzlibrary{calc}
\usetikzlibrary{calc,shapes.multipart,chains,arrows}

\setstretch{1.5}
\setlength{\parindent}{0cm}

\geometry{a4paper,top=2.5cm,bottom=2.5cm}

\titleformat{\chapter}{\Huge\Huge\bfseries}{\chaptertitlename\ \thechapter{\ }}{0pt}{\Huge}{}
\titlespacing{\chapter}{0pt}{0pt}{12pt}

\definecolor{dkgreen}{rgb}{0,0.4,0}
\definecolor{gray}{rgb}{0.5,0.5,0.5}
\definecolor{mauve}{rgb}{0.58,0,0.82}
\definecolor{LightGray}{gray}{0.9}

\lstset{
    basicstyle=\normalsize \fontspec{Consolas},    %  the size of the fonts that are used for the code
    numbers=left,            % where to put the line-numbers
    numberstyle=\color{black},  % the style that is used for the line-numbers
    numbersep=10pt,                  % how far the line-numbers are from the code
    backgroundcolor=\color{white},
    showspaces=false,
    showstringspaces=false,
    showtabs=false,
    frame=single,                   % adds a frame around the code
    rulecolor=\color{black},        % if not set, the frame-color may be changed on line-breaks within not-black text (e.g. commens (green here))
    tabsize=4,                      % sets default tabsize to 2 spaces
    captionpos=t,                   % sets the caption-position to bottom
    breaklines=false,                % sets automatic line breaking
    breakatwhitespace=true,        % sets if automatic breaks should only happen at whitespace
    title=\lstname,                   % show the filename of files included with \lstinputlisting;
    % also try caption instead of title
    numberstyle=\color{black},		% line number color
    keywordstyle=\color{blue},          % keyword style
    commentstyle=\color{dkgreen},       % comment style
    stringstyle=\color{mauve},         % string literal style
    escapeinside={\%*}{*)},            % if you want to add LaTeX within your code
    morekeywords={*,...}               % if you want to add more keywords to the set
}

\begin{document}

\pagestyle{plain}

\begin{tikzpicture}[overlay,remember picture]
    % Background color
    \fill[
        black!2]
    (current page.south west) rectangle (current page.north east);

    % Rectangles
    \shade[
        left color=Dandelion,
        right color=Dandelion!40,
        transform canvas ={rotate around ={45:($(current page.north west)+(0,-6)$)}}]
    ($(current page.north west)+(0,-6)$) rectangle ++(9,1.5);

    \shade[
        left color=lightgray,
        right color=lightgray!50,
        rounded corners=0.75cm,
        transform canvas ={rotate around ={45:($(current page.north west)+(.5,-10)$)}}]
    ($(current page.north west)+(0.5,-10)$) rectangle ++(15,1.5);

    \shade[
        left color=lightgray,
        rounded corners=0.3cm,
        transform canvas ={rotate around ={45:($(current page.north west)+(.5,-10)$)}}] ($(current page.north west)+(1.5,-9.55)$) rectangle ++(7,.6);

    \shade[
        left color=orange!80,
        right color=orange!60,
        rounded corners=0.4cm,
        transform canvas ={rotate around ={45:($(current page.north)+(-1.5,-3)$)}}]
    ($(current page.north)+(-1.5,-3)$) rectangle ++(9,0.8);

    \shade[
        left color=red!80,
        right color=red!80,
        rounded corners=0.9cm,
        transform canvas ={rotate around ={45:($(current page.north)+(-3,-8)$)}}] ($(current page.north)+(-3,-8)$) rectangle ++(15,1.8);

    \shade[
        left color=orange,
        right color=Dandelion,
        rounded corners=0.9cm,
        transform canvas ={rotate around ={45:($(current page.north west)+(4,-15.5)$)}}]
    ($(current page.north west)+(4,-15.5)$) rectangle ++(30,1.8);

    \shade[
        left color=RoyalBlue,
        right color=Emerald,
        rounded corners=0.75cm,
        transform canvas ={rotate around ={45:($(current page.north west)+(13,-10)$)}}]
    ($(current page.north west)+(13,-10)$) rectangle ++(15,1.5);

    \shade[
        left color=lightgray,
        rounded corners=0.3cm,
        transform canvas ={rotate around ={45:($(current page.north west)+(18,-8)$)}}]
    ($(current page.north west)+(18,-8)$) rectangle ++(15,0.6);

    \shade[
        left color=lightgray,
        rounded corners=0.4cm,
        transform canvas ={rotate around ={45:($(current page.north west)+(19,-5.65)$)}}]
    ($(current page.north west)+(19,-5.65)$) rectangle ++(15,0.8);

    \shade[
        left color=OrangeRed,
        right color=red!80,
        rounded corners=0.6cm,
        transform canvas ={rotate around ={45:($(current page.north west)+(20,-9)$)}}]
    ($(current page.north west)+(20,-9)$) rectangle ++(14,1.2);

    % Year
    % \draw[ultra thick,gray]
    % ($(current page.center)+(5,2)$) -- ++(0,-3cm)
    node[
            midway,
            left=0.25cm,
            text width=5cm,
            align=right,
            black!75
        ]
        {
            % {\fontsize{25}{30} \selectfont \bf ANNUAL \\[10pt] REPORT}
        }
    node[
            midway,
            right=0.25cm,
            text width=6cm,
            align=left,
            orange]
        {
            % {\fontsize{72}{86.4} \selectfont 2020}
        };

    % Title
    \node[align=center] at ($(current page.center)+(0,-5)$)
    {
    {\fontsize{72}{72} \selectfont {{离散数学}}} \\[1cm]
    {\fontsize{42}{42} \selectfont {{Discrete Mathematics}}} \\[2cm]
    {\fontsize{20}{19.2} \selectfont \textcolor{orange}{ \bf 极夜酱}} \\[4pt]
    };
\end{tikzpicture}

\newpage

\setcounter{tocdepth}{1}
\tableofcontents
\thispagestyle{empty}

\newpage

\setcounter{page}{1}

\chapter{逻辑}

\section{命题}

\subsection{命题(Proposition)}

逻辑(logic)规则给出数学语句的准确含义,这些规则用来区分有效和无效的数学论证。逻辑不仅对理解数学推理十分重要,而且在计算机科学中有许多应用,逻辑可用于电路设计、程序构造、程序正确性证明等方面。 \\

命题是逻辑的基本成分,一个命题是一个具有真值(truth value)的语句,命题可以为真也可以为假,但不能既为真又为假。 \\

\begin{table}[H]
    \centering
    \setlength{\tabcolsep}{5mm}{
        \begin{tabular}{|c|c|}
            \hline
            \textbf{命题}        & \textbf{非命题}    \\
            \hline
            I have a dog.        & What day is today? \\
            \hline
            1 + 2 = 3            & Shut the door!     \\
            \hline
            Today is Wednesday.  & 1 + 2              \\
            \hline
            It is snowing today. & x + 1 = 2          \\
            \hline
        \end{tabular}
    }
\end{table}

命题习惯上用字母$ p $, $ q $, $ r $, $ s $等来表示,如果一个命题是真命题,它的真值为真,用T表示;如果一个命题是假命题,它的真值为假,用F表示。 \\

\subsection{非运算符(NOT, Negation Operator)}

非运算符$ \neg $只作用于一个命题,其作用是反转命题的真值。 \\

真值表(truth table)可以给出命题真值之间的关系,在确定由简单命题组成的命题的真值时,真值表特别有用。

\begin{table}[H]
    \centering
    \setlength{\tabcolsep}{5mm}{
        \begin{tabular}{|c|c|}
            \hline
            \textbf{$ p $} & \textbf{$ \neg p $} \\
            \hline
            T              & F                   \\
            \hline
            F              & T                   \\
            \hline
        \end{tabular}
    }
    \caption{NOT真值表}
\end{table}

\begin{tcolorbox}
    \mybox{Exercise}
    $ \neg p $ \\
    $ p $: It snowed last night. \\
    $ \neg p $: It didn;t snow last night. \\
    $ q $: $ 2 + 3 = 6 $ \\
    $ \neg q $: $ 2 + 3 \ne 6 $
\end{tcolorbox}

\subsection{合取运算符(AND, Conjunction Operator)}

命题$ p \wedge q $表示$ p $并且$ q $,当$ p $和$ q $都为真时命题为真,否则为假。

\begin{table}[H]
    \centering
    \setlength{\tabcolsep}{5mm}{
        \begin{tabular}{|c|c|c|}
            \hline
            \textbf{$ p $} & \textbf{$ \neg p $} & \textbf{$ p \wedge q $} \\
            \hline
            T & T & T \\
            \hline
            T & F & F \\
            \hline
            F & T & F \\
            \hline
            F & F & F \\
            \hline
        \end{tabular}
    }
    \caption{AND真值表}
\end{table}

\begin{tcolorbox}
    \mybox{Exercise}
    $ p \wedge q $ \\
    $ p $: 今天是星期五。 \\
    $ q $: 今天会下雨。 \\
    $ p \wedge q $: 今天是星期五并且会下雨。
\end{tcolorbox}

\subsection{析取运算符(OR, Disjunction Operator)}

命题$ p \vee q $表示$ p $或$ q $,当$ p $和$ q $都为假时命题为假,否则为真。

\begin{table}[H]
    \centering
    \setlength{\tabcolsep}{5mm}{
        \begin{tabular}{|c|c|c|}
            \hline
            \textbf{$ p $} & \textbf{$ \neg p $} & \textbf{$ p \vee q $} \\
            \hline
            T & T & T \\
            \hline
            T & F & T \\
            \hline
            F & T & T \\
            \hline
            F & F & F \\
            \hline
        \end{tabular}
    }
    \caption{OR真值表}
\end{table}

\begin{tcolorbox}
    \mybox{Exercise}
    $ p \vee q $ \\
    $ p $: 开关坏了。 \\
    $ q $: 灯泡坏了。 \\
    $ p \vee q $: 开关坏了或者灯泡坏了。
\end{tcolorbox}

\subsection{异或运算符(XOR, Exclusive Or)}

命题$ p \oplus q $表示$ p $和$ q $的异或,当$ p $和$ q $中恰有一个为真时命题为真,否则为假。

\begin{table}[H]
    \centering
    \setlength{\tabcolsep}{5mm}{
        \begin{tabular}{|c|c|c|}
            \hline
            \textbf{$ p $} & \textbf{$ \neg p $} & \textbf{$ p \oplus q $} \\
            \hline
            T & T & F \\
            \hline
            T & F & T \\
            \hline
            F & T & T \\
            \hline
            F & F & F \\
            \hline
        \end{tabular}
    }
    \caption{XOR真值表}
\end{table}

\begin{tcolorbox}
    \mybox{Exercise}
    $ p \oplus q $ \\
    $ p $: 他现在在上海。 \\
    $ q $: 他现在在北京。 \\
    $ p \vee q $: 他现在在上海或北京。
\end{tcolorbox}

\begin{tcolorbox}
    \mybox{Exercise}
    某地发生了一件谋杀案,警察通过排查确定杀人凶手必为4个嫌疑犯的一个,根据以下信息确定凶手。 \\
    A说:不是我。 \\
    B说:是C。 \\
    C说:是D。 \\
    D说:C在胡说。 \\
    已知3个人说了真话,1个人说的是假话。
\end{tcolorbox}

\vspace{-0.5cm}
\begin{lstlisting}[language=Python]
def main():
    for killer in ['A', 'B', 'C', 'D']:
        if (killer != 'A') + (killer == 'C') \
            + (killer == 'D') + (killer != 'D') == 3:
            print(killer)

if __name__ == "__main__":
    main()
\end{lstlisting}

\begin{tcolorbox}
    \mybox{运行结果}
    C
\end{tcolorbox}

\end{document}