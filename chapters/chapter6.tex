\chapter{概率}

\section{古典概型}

\subsection{古典概型}

如果一个随机试验所包含的单位事件是有限的,且每个单位事件发生的可能性均相等,则这个随机试验叫做拉普拉斯试验,这种条件下的概率模型就叫古典概型。古典概型是概率论中最直观和最简单的模型,概率的许多运算规则,也首先是在这种模型下得到的。单位事件的特点是两两互斥的,例如抛一枚质地均匀的硬币时,正面朝上和背面朝上不会同时出现。\\

事件E是结果具有相等可能性的有限样本空间S的子集,则事件E的概率为

\vspace{-1cm}

\begin{align}
	P(E) = {|E| \over |S|}
\end{align}

\begin{tcolorbox}
	\mybox{Exercise}
	掷两个质地均匀的骰子。\\
	(1) 一共有多少种不同的结果?
	$$
		6 \times 6 = 36
	$$

	(2) 点数之和为9的结果有多少种?\\
	一共有4种:$ (3, 6), (6, 3), (4, 5), (5, 4) $\\

	(3) 点数之和为9的概率是多少?
	$$
		P(E) = {4 \over 36} = {1 \over 9}
	$$
\end{tcolorbox}

有许多试验结果的可能性并不相等,例如抛一枚不均匀的硬币,正面朝上的概率是背面朝上的两倍。

\begin{tcolorbox}
	\mybox{Exercise}
	掷一个不均匀的骰子,点数3出现的次数是其它点数的两倍,其它五个点数出现是等可能的,计算出现奇数的概率。\\
	$ E = \{1, 3, 5\} $\\
	$ P(1) = P(2) = P(4) = P(5) = P(6) = {1 \over 7} $\\
	$ P(3) = {2 \over 7} $\\
	$$
		P(E) = P(1) + P(3) + P(5) = {1 \over 7} + {2 \over 7} + {1 \over 7} = {4 \over 7}
	$$
\end{tcolorbox}

\begin{tcolorbox}
	\mybox{Exercise}
	计算从标准的52张牌中抽取5张牌,其中有4张数值相同的牌的概率。\\
	52张牌选取5张牌的组合数$ |S| = {52 \choose 5} = 2598960 $\\
	三带二的组合数$ |E| = {13 \choose 1}{4 \choose 3} \cdot {12 \choose 1}{4 \choose 2} = 3744 $\\
	$$
		P(E) = {|E| \over |S|} = {624 \over 2598960} \approx 0.00024
	$$
\end{tcolorbox}

\begin{tcolorbox}
	\mybox{Exercise}
	计算从标准的52张牌中抽取的5张牌为三带二的概率。\\
	52张牌选取5张牌的组合数$ |S| = {52 \choose 5} = 2598960 $\\
	4张牌数值相同的组合数$ |E| = {13 \choose 1}{4 \choose 4}{48 \choose 1} = 624 $,其中$ 13 \choose 1 $表示选择一种数值的方式数,$ 4 \choose 4 $表示从4种花色中选出4张该数值的方式数,$ 48 \choose 1 $表示选择第5张牌的方式数。\\
	$$
		P(E) = {|E| \over |S|} = {3744 \over 2598960} \approx 0.0014
	$$
\end{tcolorbox}

\vspace{0.5cm}

\subsection{有放回/无放回抽样}

\begin{tcolorbox}
	\mybox{Exercise}
	计算从50个从1开始标号的球中,依次取出号码为18、22、21、1、49的球的概率。\\
	(a) 每次取完球后,不把球放回箱子。\\
	$$
		P(E) = {1 \over P(50, 5)} = {1 \over {50 \cdot 49 \cdot 48 \cdot 48 \cdot 47 \cdot 46}} = 254251200
	$$
	\\
	(b) 每次取完球后,把球放回箱子。\\
	$$
		P(E) = {1 \over 50^5} = 312500000
	$$
\end{tcolorbox}

\vspace{0.5cm}

\subsection{对立事件概率}

假设$ E $是样本空间$ S $的一个事件,事件$ \overline{E} = S - E $的概率是

\vspace{-1cm}

\begin{align}
	P(\overline{E}) = 1 - P(E)
\end{align}

\begin{tcolorbox}
	\mybox{Exercise}
	随机生成一个10位的二进制串,计算其中至少包含2个0的概率。\\
	没有0的组合数:$ 10 \choose 0 $\\
	1个0的组合数:$ 10 \choose 1$
	\begin{align*}
		P(E) & = 1 - P(\overline{{E}})                                \\
		     & = 1 - {{10 \choose 0} + {10 \choose 1} \over {2^{10}}} \\
		     & = 1 - {11 \over 1024}                                  \\
		     & = {1013 \over 1024}
	\end{align*}
\end{tcolorbox}

\newpage

\section{概率推理}

\subsection{三门问题(Monty Hall Problem)}

三门问题也叫蒙提霍尔问题,是一个有关于博弈论的数学问题。美国主持人Monty Hall主持了一档电视游戏节目,在节目中有三扇门,这三扇门的后面分别会被随机地放进汽车和两只山羊。参赛者要随机选择一扇门,在参赛者选择了一扇门之后,主持人并不会立刻打开这扇门,而是从剩下的两扇门中打开一扇有山羊的门。随后主持人会给竞猜者提供一次重新选择门的机会,此时竞猜者可以保持自己的第一选择不变,也可以更换自己的选择。那么参赛者到底是应该怎么做能让得到汽车大奖的概率大一些呢?\\

该节目播出之后,引来了大家的热议。一名杂志专栏作者Marilyn Vos Savant曾先后三次对此问题作答,试图说服读者相信改变选择对参赛者是有利的,也就是换门比不换门得到汽车的概率要大一些,前者概率为$ 2 \over 3 $,后者概率为$ 1 \over 3 $。然而她的这一观点提出之后,绝大多数的读者都不接受她给出的答案。人们寄来了数千封抱怨信,很多寄信人是老师或者学者。\\

一位来自University of Florida的读者写道:“这个国家已经有够多的数学文盲了,我们不想再有个世界上智商最高的人来充数!真让人羞愧!”\\

另一个人写道:“我看你就是那只山羊!”\\

美国陆军研究所的Everett Harman写道:“如果连博士都要出错,我看这个国家马上要陷入严重的麻烦了。”\\

时至今日,对于三门问题的争论仍在继续。对于这个问题有两种观点,一种认为改不改变获得汽车的概率都是一样的,都是$ 1 \over 2 $;另一种观点则认为改变选择获得汽车的概率是$ 2 \over 3 $,而不改变选择的到汽车的概率只有$ 1 \over 3 $。持第二种观点的人认为,在这个游戏的过程当中,整体是一个关于条件概率的两个阶段的决策问题,也就是说,起初选择一扇认为有奖品的门,在主持人开了一扇没有奖品的门之后,对于参赛者要选择坚持最初的还是改变最初的选择是一个连续的动作,那么利用条件概率的贝叶斯定理就可以说明改变选择会使得到汽车的概率增加。\\

随后,MIT的数学家和阿拉莫斯国家实验室的程序员都宣布,他们用计算机进行模拟实验的结果,支持了Marilyn的答案。\\

可以看出,这是一个概率论和人的直觉不太符合的例子。这告诉我们在做基于量化的判断的时候,要以事实和数据为依据,而不要凭主观和直觉来决定。\\

那么正确结果$ 2 \over 3 $是怎么来的呢?其中有一个非常重要隐藏条件,作为知道答案的主持人,不可能选择开启有车的门,所以他永远都会挑一扇有山羊的门,也就是说主持人选择开启其中一扇门时,他的选择并不是一个纯随机事件。\\

遍历所有可能性:

\begin{itemize}
	\item 如果参赛者选择汽车,主持人选择山羊1,改变选择不中奖。
	\item 如果参赛者选择山羊1,主持人选择山羊2,改变选择中奖。
	\item 如果参赛者选择山羊2,主持人选择山羊1,改变选择中奖。
\end{itemize}

因此可见改变选择后的成功概率为$ 2 \over 3 $。\\

延伸一下,可以看个类似的情况。54张牌,抽中大王算赢,参赛者选中一张后先不要看,主持人去除53张牌里的52张非大王的牌,问你拿手里的牌换剩余的一张牌,是否能提高胜率?

\newpage

\section{概率论}

\subsection{条件概率}

