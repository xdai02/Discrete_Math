\chapter{概率}

\section{古典概型}

\subsection{古典概型}

如果一个随机试验所包含的单位事件是有限的,且每个单位事件发生的可能性均相等,则这个随机试验叫做拉普拉斯试验,这种条件下的概率模型就叫古典概型。古典概型是概率论中最直观和最简单的模型,概率的许多运算规则,也首先是在这种模型下得到的。单位事件的特点是两两互斥的,例如抛一枚质地均匀的硬币时,正面朝上和背面朝上不会同时出现。\\

事件E是结果具有相等可能性的有限样本空间S的子集,则事件E的概率为

\vspace{-1cm}

\begin{align}
	P(E) = {|E| \over |S|}
\end{align}

\begin{tcolorbox}
	\mybox{Exercise}
	掷两个质地均匀的骰子。\\
	(1) 一共有多少种不同的结果?
	$$
		6 \times 6 = 36
	$$

	(2) 点数之和为9的结果有多少种?\\
	一共有4种:$ (3, 6), (6, 3), (4, 5), (5, 4) $\\

	(3) 点数之和为9的概率是多少?
	$$
		P(E) = {4 \over 36} = {1 \over 9}
	$$
\end{tcolorbox}

\begin{tcolorbox}
	\mybox{Exercise}
	计算从标准的52张牌中抽取5张牌,其中有4张数值相同的牌的概率。\\
	52张牌选取5张牌的组合数$ |S| = {52 \choose 5} = 2598960 $\\
	三带二的组合数$ |E| = {13 \choose 1}{4 \choose 3} \cdot {12 \choose 1}{4 \choose 2} = 3744 $\\
	$$
		P(E) = {|E| \over |S|} = {624 \over 2598960} \approx 0.00024
	$$
\end{tcolorbox}

\begin{tcolorbox}
	\mybox{Exercise}
	计算从标准的52张牌中抽取的5张牌为三带二的概率。\\
	52张牌选取5张牌的组合数$ |S| = {52 \choose 5} = 2598960 $\\
	4张牌数值相同的组合数$ |E| = {13 \choose 1}{4 \choose 4}{48 \choose 1} = 624 $,其中$ 13 \choose 1 $表示选择一种数值的方式数,$ 4 \choose 4 $表示从4种花色中选出4张该数值的方式数,$ 48 \choose 1 $表示选择第5张牌的方式数。\\
	$$
		P(E) = {|E| \over |S|} = {3744 \over 2598960} \approx 0.0014
	$$
\end{tcolorbox}

\vspace{0.5cm}

\subsection{有放回/无放回抽样}

\begin{tcolorbox}
	\mybox{Exercise}
	计算从50个从1开始标号的球中,依次取出号码为18、22、21、1、49的球的概率。\\
	(a) 每次取完球后,不把球放回箱子。\\
	$$
		P(E) = {1 \over P(50, 5)} = {1 \over {50 \cdot 49 \cdot 48 \cdot 48 \cdot 47 \cdot 46}} = 254251200
	$$
	\\
	(b) 每次取完球后,把球放回箱子。\\
	$$
		P(E) = {1 \over 50^5} = 312500000
	$$
\end{tcolorbox}

\vspace{0.5cm}

\subsection{对立事件概率}

