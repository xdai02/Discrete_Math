\chapter{概率}

\section{古典概型}

\subsection{古典概型}

如果一个随机试验所包含的单位事件是有限的,且每个单位事件发生的可能性均相等,则这个随机试验叫做拉普拉斯试验,这种条件下的概率模型就叫古典概型。古典概型是概率论中最直观和最简单的模型,概率的许多运算规则,也首先是在这种模型下得到的。\\

单位事件的特点是两两互斥的,例如抛一枚质地均匀的硬币时,正面朝上和背面朝上不会同时出现。\\

在古典概型中,概率的计算公式为:

$$
	P(A) = {A\text{包含的单位事件个数}m \over \text{单位事件的总数}n}
$$

\begin{tcolorbox}
	\mybox{Exercise}
	掷两个质地均匀的骰子。\\
	(1) 一共有多少种不同的结果?
	$$
		6 \times 6 = 36
	$$

	(2) 点数之和为9的结果有多少种?\\
	一共有4种:$ (3, 6), (6, 3), (4, 5), (5, 4) $\\

	(3) 点数之和为9的概率是多少?
	$$
		P(A) = {4 \over 36} = {1 \over 9}
	$$
\end{tcolorbox}