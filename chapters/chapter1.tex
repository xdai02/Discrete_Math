\chapter{逻辑}

\section{命题}

\subsection{命题(Proposition)}

逻辑(logic)规则给出数学语句的准确含义,这些规则用来区分有效和无效的数学论证。逻辑不仅对理解数学推理十分重要,而且在计算机科学中有许多应用,逻辑可用于电路设计、程序构造、程序正确性证明等方面。\\

命题是逻辑的基本成分,一个命题是一个具有真值(truth value)的语句,命题可以为真也可以为假,但不能既为真又为假。

\begin{table}[H]
	\centering
	\setlength{\tabcolsep}{5mm}{
		\begin{tabular}{|c|c|}
			\hline
			\textbf{命题}        & \textbf{非命题}    \\
			\hline
			I have a dog.        & What day is today? \\
			\hline
			1 + 2 = 3            & Shut the door!     \\
			\hline
			Today is Wednesday.  & 1 + 2              \\
			\hline
			It is snowing today. & x + 1 = 2          \\
			\hline
		\end{tabular}
	}
	\caption{命题与非命题}
\end{table}

命题习惯上用字母$ p $, $ q $, $ r $, $ s $等来表示,如果一个命题是真命题,它的真值为真,用T表示;如果一个命题是假命题,它的真值为假,用F表示。\\

\subsection{非运算符(NOT, Negation Operator)}

非运算符$ \neg $只作用于一个命题,其作用是反转命题的真值。\\

真值表(truth table)可以给出命题真值之间的关系,在确定由简单命题组成的命题的真值时,真值表特别有用。

\begin{table}[H]
	\centering
	\setlength{\tabcolsep}{5mm}{
		\begin{tabular}{|c|c|}
			\hline
			\textbf{$ p $} & \textbf{$ \neg p $} \\
			\hline
			T              & F                   \\
			\hline
			F              & T                   \\
			\hline
		\end{tabular}
	}
	\caption{NOT真值表}
\end{table}

\begin{tcolorbox}
	\mybox{Exercise}
	$ \neg p $\\
	$ p $: It snowed last night.\\
	$ \neg p $: It didn;t snow last night.\\
	$ q $: $ 2 + 3 = 6 $\\
	$ \neg q $: $ 2 + 3 \ne 6 $
\end{tcolorbox}

\vspace{0.5cm}

\subsection{合取运算符(AND, Conjunction Operator)}

命题$ p \wedge q $表示$ p $并且$ q $,当$ p $和$ q $都为真时命题为真,否则为假。

\begin{table}[H]
	\centering
	\setlength{\tabcolsep}{5mm}{
		\begin{tabular}{|c|c|c|}
			\hline
			\textbf{$ p $} & \textbf{$ q $} & \textbf{$ p \wedge q $} \\
			\hline
			T              & T              & T                       \\
			\hline
			T              & F              & F                       \\
			\hline
			F              & T              & F                       \\
			\hline
			F              & F              & F                       \\
			\hline
		\end{tabular}
	}
	\caption{AND真值表}
\end{table}

\begin{tcolorbox}
	\mybox{Exercise}
	$ p \wedge q $\\
	$ p $: 今天是星期五。\\
	$ q $: 今天会下雨。\\
	$ p \wedge q $: 今天是星期五并且会下雨。
\end{tcolorbox}

\vspace{0.5cm}

\subsection{析取运算符(OR, Disjunction Operator)}

命题$ p \vee q $表示$ p $或$ q $,当$ p $和$ q $都为假时命题为假,否则为真。

\begin{table}[H]
	\centering
	\setlength{\tabcolsep}{5mm}{
		\begin{tabular}{|c|c|c|}
			\hline
			\textbf{$ p $} & \textbf{$ q $} & \textbf{$ p \vee q $} \\
			\hline
			T              & T              & T                     \\
			\hline
			T              & F              & T                     \\
			\hline
			F              & T              & T                     \\
			\hline
			F              & F              & F                     \\
			\hline
		\end{tabular}
	}
	\caption{OR真值表}
\end{table}

\begin{tcolorbox}
	\mybox{Exercise}
	$ p \vee q $\\
	$ p $: 开关坏了。\\
	$ q $: 灯泡坏了。\\
	$ p \vee q $: 开关坏了或者灯泡坏了。
\end{tcolorbox}

\vspace{0.5cm}

\subsection{异或运算符(XOR, Exclusive Or)}

命题$ p \oplus q $表示$ p $和$ q $的异或,当$ p $和$ q $中恰有一个为真时命题为真,否则为假。

\begin{table}[H]
	\centering
	\setlength{\tabcolsep}{5mm}{
		\begin{tabular}{|c|c|c|}
			\hline
			\textbf{$ p $} & \textbf{$ q $} & \textbf{$ p \oplus q $} \\
			\hline
			T              & T              & F                       \\
			\hline
			T              & F              & T                       \\
			\hline
			F              & T              & T                       \\
			\hline
			F              & F              & F                       \\
			\hline
		\end{tabular}
	}
	\caption{XOR真值表}
\end{table}

\begin{tcolorbox}
	\mybox{Exercise}
	$ p \oplus q $\\
	$ p $: 他现在在上海。\\
	$ q $: 他现在在北京。\\
	$ p \vee q $: 他现在在上海或北京。
\end{tcolorbox}

\begin{tcolorbox}
	\mybox{Exercise}
	某地发生了一件谋杀案,警察通过排查确定杀人凶手必为4个嫌疑犯的一个,根据以下信息确定凶手。\\
	A说:不是我。\\
	B说:是C。\\
	C说:是D。\\
	D说:C在胡说。\\
	已知3个人说了真话,1个人说的是假话。
\end{tcolorbox}

\begin{lstlisting}[language=Python]
def main():
    for killer in ['A', 'B', 'C', 'D']:
        if (killer != 'A') + (killer == 'C') \
            + (killer == 'D') + (killer != 'D') == 3:
            print(killer)

if __name__ == "__main__":
    main()
\end{lstlisting}

\begin{tcolorbox}
	\mybox{运行结果}
	C
\end{tcolorbox}

\newpage

\section{复合命题}

\subsection{复合命题(Compound Proposition)}

使用非运算符和已定义的各联结词可以构造复合命题。小括号用于规定复合命题中多个逻辑运算符的操作顺序,为了减少所需的小括号数量,规定了各联结词的优先级。

\begin{table}[H]
	\centering
	\setlength{\tabcolsep}{5mm}{
		\begin{tabular}{|c|c|}
			\hline
			\textbf{优先级} & \textbf{运算符}                   \\
			\hline
			1               & $ \neg $                          \\
			\hline
			2               & $ \wedge $ / $ \vee $             \\
			\hline
			3               & $ \rightarrow / \leftrightarrow $ \\
			\hline
		\end{tabular}
	}
	\caption{运算符优先级}
\end{table}

\vspace{0.5cm}

\subsection{蕴含运算符(Implication Operator)}

命题$ p \rightarrow q $表示$ p $蕴含$ q $,在$ p $为真而$ q $为假时命题为假,否则为真。其中$ p $称为前提,$ q $称为结论。

\begin{table}[H]
	\centering
	\setlength{\tabcolsep}{5mm}{
		\begin{tabular}{|c|c|c|}
			\hline
			\textbf{$ p $} & \textbf{$ q $} & \textbf{$ p \rightarrow q $} \\
			\hline
			T              & T              & T                            \\
			\hline
			T              & F              & F                            \\
			\hline
			F              & T              & T                            \\
			\hline
			F              & F              & T                            \\
			\hline
		\end{tabular}
	}
	\caption{蕴含真值表}
\end{table}

表示$ p \rightarrow q $的术语有很多种,常见的有:

\begin{itemize}
	\item If $ p $, then $ q $.
	\item $ p $ only if $ q $.
	\item $ q $ is necessary for $ p $.
\end{itemize}

\begin{tcolorbox}
	\mybox{Exercise}
	$ p \rightarrow q $\\
	$ p $:我去看电影。\\
	$ q $:我买奶茶。\\
	$ p \rightarrow q $:如果我去看电影,那么我会买奶茶。
\end{tcolorbox}

\begin{figure}[H]
	\centering
	\includegraphics[scale=0.7]{img/C1/1-2/1.png}
\end{figure}

由$ p \rightarrow q $可以构造出几个相关的蕴含:

\begin{enumerate}
	\item $ q \rightarrow p $称为$ p \rightarrow q $的逆命题(converse)。
	\item $ \neg q \rightarrow \neg p $称为$ p \rightarrow q $的逆否命题(contrapositive)。
\end{enumerate}

\begin{tcolorbox}
	\mybox{Exercise}
	逆命题与逆否命题\\
	$ p $:今天是星期四。\\
	$ q $:我今天有考试。\\
	$ p \rightarrow q $:如果今天是星期四,那么我今天有考试。\\
	$ q \rightarrow p $:如果我今天有考试,那么果今天是星期四。\\
	$ \neg q \rightarrow \neg p $:如果我今天没有考试,那么今天不是星期四。
\end{tcolorbox}

\vspace{0.5cm}

\subsection{双向蕴含(Biconditional Operation)}

命题$ p \leftrightarrow q $表示$ p $双向蕴含$ q $,在$ p $和$ q $有相同的真值时命题为真,否则为假。

\begin{table}[H]
	\centering
	\setlength{\tabcolsep}{5mm}{
		\begin{tabular}{|c|c|c|}
			\hline
			\textbf{$ p $} & \textbf{$ q $} & \textbf{$ p \leftrightarrow q $} \\
			\hline
			T              & T              & T                                \\
			\hline
			T              & F              & F                                \\
			\hline
			F              & T              & F                                \\
			\hline
			F              & F              & T                                \\
			\hline
		\end{tabular}
	}
	\caption{双向蕴含真值表}
\end{table}

双蕴含的真值表与异或的真值表正好相反,因此$ p \leftrightarrow q $与$ \neg (p \oplus q) $等价。

\newpage

\section{逻辑等价}

\subsection{逻辑等价(Logical Equivalence)}

两个不同的复合命题可能拥有完全相同的真值,则称这两个命题在逻辑上是等价的。如果无论复合命题中各个命题的真值是什么,复合命题的真值总是为真,这样的复合命题称为永真式(tautology)。如果真值永远为假的复合命题称为矛盾(contradiction)。

\begin{table}[H]
	\centering
	\setlength{\tabcolsep}{5mm}{
		\begin{tabular}{|c|c|c|c|}
			\hline
			\textbf{$ p $} & \textbf{$ \neg p $} & \textbf{$ p \vee \neg p $} & \textbf{$ p \wedge \neg p $} \\
			\hline
			T              & F                   & T                          & F                            \\
			\hline
			F              & T                   & T                          & F                            \\
			\hline
		\end{tabular}
	}
	\caption{逻辑等价}
\end{table}

如果复合命题$ s $和是$ r $逻辑等价的,可表示为$ s \equiv r $。只有当$ s \leftrightarrow r $是永真式时,$ s $和$ r $才是逻辑等价的。

\begin{tcolorbox}
	\mybox{Exercise}
	使用真值表证明$ p \vee q \equiv \neg (\neg p \wedge \neg q) $
	\begin{table}[H]
		\centering
		\setlength{\tabcolsep}{5mm}{
			\begin{tabular}{|c|c|c|c|c|c|c|}
				\hline
				\textbf{$ p $} & \textbf{$ q $} & \textbf{$ p \vee q $} & \textbf{$ \neg p $} & \textbf{$ \neg q $} & \textbf{$ \neg p \wedge \neg q $} & \textbf{$ \neg (\neg p \wedge \neg q) $} \\
				\hline
				T              & T              & T                     & F                   & F                   & F                                 & T                                        \\
				\hline
				T              & F              & T                     & F                   & T                   & F                                 & T                                        \\
				\hline
				F              & T              & T                     & T                   & F                   & F                                 & T                                        \\
				\hline
				F              & F              & F                     & T                   & T                   & T                                 & F                                        \\
				\hline
			\end{tabular}
		}
	\end{table}
\end{tcolorbox}

\vspace{0.5cm}

\subsection{逻辑等价定理}

\begin{tcolorbox}
	\mybox{幂等律 Idempotent Laws}
	\begin{align}
		p \wedge p & \equiv p \\
		p \vee p   & \equiv p
	\end{align}
\end{tcolorbox}

\begin{tcolorbox}
	\mybox{恒等律 Identity Laws}
	\begin{align}
		p \wedge T & \equiv p \\
		p \vee F   & \equiv p
	\end{align}
\end{tcolorbox}

\begin{tcolorbox}
	\mybox{支配律 Domination Laws}
	\begin{align}
		p \vee T   & \equiv T \\
		p \wedge F & \equiv F
	\end{align}
\end{tcolorbox}

\begin{tcolorbox}
	\mybox{双非律 Double Negation Law}
	\begin{align}
		\neg (\neg p) & \equiv p
	\end{align}
\end{tcolorbox}

\begin{tcolorbox}
	\mybox{交换律 Commutative Laws}
	\begin{align}
		p \wedge q & \equiv q \wedge p \\
		p \vee q   & \equiv q \vee p
	\end{align}
\end{tcolorbox}

\begin{tcolorbox}
	\mybox{结合律 Associative Laws}
	\begin{align}
		(p \wedge q) \wedge r & \equiv p \wedge (q \wedge r) \\
		(p \vee q) \vee r     & \equiv p \vee (q \vee r)
	\end{align}
\end{tcolorbox}

\begin{tcolorbox}
	\mybox{分配律 Distributive Laws}
	\begin{align}
		(p \wedge q) \wedge r & \equiv p \wedge (q \wedge r) \\
		(p \vee q) \vee r     & \equiv p \vee (q \vee r)
	\end{align}
\end{tcolorbox}

\begin{tcolorbox}
	\mybox{德摩根律 De Morgan's Laws}
	\begin{align}
		\neg (p \wedge q) & \equiv \neg p \vee \neg q   \\
		\neg (p \vee q)   & \equiv \neg p \wedge \neg q
	\end{align}
\end{tcolorbox}

\begin{tcolorbox}
	\mybox{吸收律 Absorption Laws}
	\begin{align}
		p \wedge (p \vee q) & \equiv p \\
		p \vee (p \wedge q) & \equiv p
	\end{align}
\end{tcolorbox}

\begin{tcolorbox}
	\mybox{条件恒等}
	\begin{align}
		p \rightarrow q     & \equiv \neg p \vee q                              \\
		p \leftrightarrow q & \equiv (p \rightarrow q) \wedge (q \rightarrow p)
	\end{align}
\end{tcolorbox}

\begin{tcolorbox}
	\mybox{Exercise}
	证明$ (p \vee q) \rightarrow p $永真
	\begin{align*}
		 & (p \vee q) \rightarrow p           \\
		 & \equiv \neg (p \wedge q) \vee p    \\
		 & \equiv (\neg p \vee \neg q) \vee p \\
		 & \equiv (\neg q \vee \neg p) \vee p \\
		 & \equiv \neg q \vee (\neg p \vee p) \\
		 & \equiv \neg q \vee T               \\
		 & \equiv T
	\end{align*}
\end{tcolorbox}

\newpage

\section{谓词与量词}

\subsection{谓词(Predicate)}

命题逻辑并不能表达数学语言和自然语言中所有语句的确切含义。在数学表达式和计算机程序中经常可以看到含有变量的语句,例如$ x > 3 $、$ x = y + 3 $、程序$ x $正在运行等。当变量值未指定时,这些语句既不为真也不为假。\\

利用$ P(x) $可以表示语句,其中$ x $是变量,语句$ P(x) $可以说是命题函数$ P $在$ x $的值。一旦给变量$ x $赋一个值,语句$ P(x) $就称为命题并具有真值。\\

通常使用大写字母$ P $, $ Q $, $ R $等表示谓词,小写字母$ x $, $ y $, $ z $等表示变量。

\begin{tcolorbox}
	\mybox{Exercise}
	谓词
	\begin{table}[H]
		\centering
		\setlength{\tabcolsep}{5mm}{
			\begin{tabular}{|l|l|}
				\hline
				\textbf{谓词}          & \textbf{真值}      \\
				\hline
				$ P(x): x + 3 = 6 $    & $ P(3) $为True     \\
				\hline
				$ Q(x, y): x = y + 2 $ & $ Q(4, 1) $为False \\
				\hline
			\end{tabular}
		}
	\end{table}
\end{tcolorbox}

\vspace{0.5cm}

\subsection{量词(Quantifier)}

量词用量化表示在何种程度上谓词对于一定范围的个体成立。量词有全称量词(universal quantifier)和存在量词(existential quantifer)。\\

全称量词$ \forall $表示all。$ \forall_x P(x) $是一个命题,当范围内所有的$ x $都能使语句$ P(x) $为真时,命题为真。

\vspace{-0.5cm}

$$
	\forall_x P(x) = P(a_1) \wedge P(a_2) \wedge \dots \wedge P(a_k)
$$

\begin{tcolorbox}
	\mybox{Exercise}
	全称量词\\
	假设$ x $表示全班所有学生,$ P(x) $表示$ x $完成了作业。\\
	$ \forall_x P(x) $:全班所有学生都完成了作业。
\end{tcolorbox}

存在量词$ \exists $表示exists。$ \exists_x P(x) $是一个命题,当范围内存在至少一个$ x $能够语句$ P(x) $为真时,命题为真。

\vspace{-0.5cm}

$$
	\exists_x P(x) = P(a_1) \vee P(a_2) \vee \dots \vee P(a_k)
$$

\begin{tcolorbox}
	\mybox{Exercise}
	存在量词\\
	假设$ x $表示全班所有学生,$ P(x) $表示$ x $完成了作业。\\
	$ \exists_x P(x) $:班里存在有一个学生完成了作业。
\end{tcolorbox}

\begin{tcolorbox}
	\mybox{Exercise}
	嵌套量词\\
	假设$ x $表示某个人,$ P(x) $表示有父母。\\
	$ \forall_x P(x) $:所有人都有父母。\\
	$ \exists_x \neg P(x) $:存在至少有一个人没有父母。\\
	$ \exists_x \exists_y (P(x) \wedge P(y)) $:至少存在一个人$ x $和一个人$ y $有父母。
\end{tcolorbox}

\begin{tcolorbox}
	\mybox{Exercise}
	$ P(x) $:$ x $是偶数,$ Q(x) $:$ x $能被3整除,$ x \in \mathbb{Z}^+ $
	\begin{table}[H]
		\centering
		\setlength{\tabcolsep}{5mm}{
			\begin{tabular}{|l|l|}
				\hline
				\textbf{语句}                              & \textbf{真值} \\
				\hline
				$ \exists_x (P(x) \wedge Q(x)) $           & True          \\
				\hline
				$ \forall_x (P(x) \rightarrow \neg Q(x)) $ & False         \\
				\hline
			\end{tabular}
		}
	\end{table}
\end{tcolorbox}

\vspace{0.5cm}

\subsection{全称量词的否定}

否定运算符可以使用在全称量词上。

\vspace{-1cm}

\begin{align}
	\neg \forall_x P(x) & \equiv \exists_x \neg P(x) \\
	\neg \exists_x P(x) & \equiv \forall_x \neg P(x)
\end{align}

\begin{tcolorbox}
	\mybox{Exercise}
	全称量词的否定\\
	$ P(x) $: $ x $ will pass the course ($ x $ is a student).\\
	$ \neg \forall_x P(x) $: Not all students will pass the course.\\
	$ \forall_x \neg P(x) $: No student will pass the course.\\
	$ \neg \exists_x P(x) $: There does not exist a student that will pass the course.\\
	$ \exists_x \neg P(x) $: There exists a student that will not pass the course.
\end{tcolorbox}

\newpage

\section{证明}

\subsection{证明(Proof)}

证明方法非常重要,不仅因为它们可用于证明数学定理,而且在计算机科学中也有许多应用,包括验证程序正确性、建立安全的操作系统、人工智能领域做推论等。证明就是建立定理真实性的有效论证。\\

证明定理有很多方法:

\begin{enumerate}
	\item 直接证明法(direct proof)

	      \begin{tcolorbox}
		      \mybox{证明}
		      如果$ n $是奇数,那么$ n^2 $也是奇数,$ n \in \mathbb{Z} $。
		      \begin{align*}
			      n^2 & = (2k + 1)^2       \\
			          & = 4k^2 + 4k + 1    \\
			          & = 2(2k^2 + 2k) + 1
		      \end{align*}
	      \end{tcolorbox}

	\item 反证法(proof by contrapositive):由于$ p \rightarrow q \equiv \neg q \rightarrow \neg p $,因此可以通过证明原命题的逆否命题来反证原命题。

	      \begin{tcolorbox}
		      \mybox{证明}
		      如果$ xy $是偶数,那么$ x $是偶数或$ y $是偶数,$ x, y \in \mathbb{Z} $。\\
		      逆否命题:如果$ x $是奇数并且$ y $是奇数,那么$ xy $是奇数,$ x, y \in \mathbb{Z} $。
		      \begin{align*}
			      xy & = (2m + 1)(2n + 1)   \\
			         & = 4mn + 2m +2n + 1   \\
			         & = 2(2mn + m + n) + 1
		      \end{align*}
	      \end{tcolorbox}
\end{enumerate}

\newpage

\section{布尔代数}

\subsection{布尔代数(Boolean Algebra)}

计算机和其它电子设备中的电路都有输入和输出,输入是0或1,输出也是0或1。电路可以用任何具有两个不同状态的基本元件来构造,例如开关和光学装置就是这样的原件,开关可位于开或关的位置,光学装置可能是点亮或未点亮。18世纪,乔治·布尔(George Boole)给出了逻辑的基本规则。\\

电路的操作可以用布尔函数来定义,这样的布尔函数对任意一组输入都能指出其输出的值。\\

布尔代数提供的是集合$ \{0, 1\} $上的运算和规则,布尔代数的规则类似于命题逻辑的规则。1相当于逻辑中的真,0相当于逻辑中的假。\\

布尔代码运算主要有三种:

\begin{enumerate}
	\item 补(complement)
	      \begin{table}[H]
		      \centering
		      \setlength{\tabcolsep}{5mm}{
			      \begin{tabular}{|c|c|}
				      \hline
				      \textbf{$ x $} & \textbf{$ \overline{x} $} \\
				      \hline
				      1              & 0                         \\
				      \hline
				      0              & 1                         \\
				      \hline
			      \end{tabular}
		      }
	      \end{table}

	\item 布尔积(boolean multiplication)
	      \begin{table}[H]
		      \centering
		      \setlength{\tabcolsep}{5mm}{
			      \begin{tabular}{|c|c|c|}
				      \hline
				      \textbf{$ x $} & \textbf{$ y $} & \textbf{$ x \cdot y $} \\
				      \hline
				      1              & 1              & 1                      \\
				      \hline
				      1              & 0              & 0                      \\
				      \hline
				      0              & 1              & 0                      \\
				      \hline
				      0              & 0              & 0                      \\
				      \hline
			      \end{tabular}
		      }
	      \end{table}

	\item 布尔和(boolean addition)
	      \begin{table}[H]
		      \centering
		      \setlength{\tabcolsep}{5mm}{
			      \begin{tabular}{|c|c|c|}
				      \hline
				      \textbf{$ x $} & \textbf{$ y $} & \textbf{$ x + y $} \\
				      \hline
				      1              & 1              & 1                  \\
				      \hline
				      1              & 0              & 1                  \\
				      \hline
				      0              & 1              & 1                  \\
				      \hline
				      0              & 0              & 0                  \\
				      \hline
			      \end{tabular}
		      }
	      \end{table}
\end{enumerate}

\begin{tcolorbox}
	\mybox{Exercise}
	当$ x = y = 1 $,$ w = z = 0 $时,\\
	$ x \cdot y + (w + z) = 1 \cdot 1 + (0 + 0) = 1 + 0 = 1 $\\
	$ x \cdot \overline y + z \cdot \overline{(w + z)} = 1 \cdot \overline 1 + 0 \cdot \overline{(0 + 0)} = 0 + 0 = 0 $\\
	$ x \cdot \overline y + \overline{(\overline x + y + \overline{yz})} = 1 \cdot \overline 1 + \overline{(\overline 1 + 1 + \overline{1 \cdot 0})} = 0 + \overline 1 = 0 $
\end{tcolorbox}

\vspace{0.5cm}

\subsection{布尔代数定理}

\begin{tcolorbox}
	\mybox{幂等律 Idempotent Laws}
	\begin{align}
		x \cdot x = 0 \\
		x + x = x
	\end{align}
\end{tcolorbox}

\begin{tcolorbox}
	\mybox{恒等律 Identity Laws}
	\begin{align}
		x \cdot 1 = x \\
		x + 0 = x
	\end{align}
\end{tcolorbox}

\begin{tcolorbox}
	\mybox{支配律 Domination Laws}
	\begin{align}
		x \cdot 0 = 0 \\
		x + 1 = 1
	\end{align}
\end{tcolorbox}

\begin{tcolorbox}
	\mybox{双非律 Double Negation Law}
	\begin{align}
		\overline {\overline x} = x
	\end{align}
\end{tcolorbox}

\begin{tcolorbox}
	\mybox{交换律 Commutative Laws}
	\begin{align}
		x \cdot y = y \cdot x \\
		x + y = y + x
	\end{align}
\end{tcolorbox}

\begin{tcolorbox}
	\mybox{结合律 Associative Laws}
	\begin{align}
		(x \cdot y) \cdot z = x \cdot (y \cdot z) \\
		(x + y) + z = x + (y + z)
	\end{align}
\end{tcolorbox}

\begin{tcolorbox}
	\mybox{分配律 Distributive Laws}
	\begin{align}
		x \cdot (y + z) = x \cdot y + x \cdot z \\
		x + (y \cdot z) = (x + y) \cdot (x + z)
	\end{align}
\end{tcolorbox}

\begin{tcolorbox}
	\mybox{德摩根律 De Morgan's Laws}
	\begin{align}
		\overline{x \cdot y} = \overline x + \overline y \\
		\overline{x + y} = \overline x \cdot \overline y
	\end{align}
\end{tcolorbox}

\begin{tcolorbox}
	\mybox{吸收律 Absorption Laws}
	\begin{align}
		x \cdot (x + y) = x \\
		x + (x \cdot y) = x
	\end{align}
\end{tcolorbox}

\begin{tcolorbox}
	\mybox{证明}
	$ xy + x \overline y = x $
	\begin{align}
		 & xy + x \overline y          \\
		 & = x \cdot (y + \overline y) \\
		 & = x \cdot 1                 \\
		 & = x
	\end{align}
\end{tcolorbox}

\vspace{0.5cm}

\subsection{布尔函数(Boolean Function)}

含有$ n $个变量的布尔函数能够构造出$ 2^n $行的输入输出表。

\begin{tcolorbox}
	\mybox{Exercise}
	计算$ F(x, y, z) = xy + \overline z $
	\begin{table}[H]
		\centering
		\setlength{\tabcolsep}{5mm}{
			\begin{tabular}{|c|c|c|c|}
				\hline
				\textbf{$ x $} & \textbf{$ y $} & \textbf{$ z $} & \textbf{$ F(x, y, z) $} \\
				\hline
				0              & 0              & 0              & 1                       \\
				\hline
				0              & 0              & 1              & 0                       \\
				\hline
				0              & 1              & 0              & 1                       \\
				\hline
				0              & 1              & 1              & 0                       \\
				\hline
				1              & 0              & 0              & 1                       \\
				\hline
				1              & 0              & 1              & 0                       \\
				\hline
				1              & 1              & 0              & 1                       \\
				\hline
				1              & 1              & 1              & 1                       \\
				\hline
			\end{tabular}
		}
	\end{table}
\end{tcolorbox}

\vspace{0.5cm}

\subsection{NAND与NOR}

NAND运算符用$ \uparrow $表示Not And:

\vspace{-0.5cm}

$$
	x \uparrow y = \overline{x \cdot y}
$$

NOR运算符用$ \downarrow $表示Not Or:

\vspace{-0.5cm}

$$
	x \downarrow y = \overline{x + y}
$$

\newpage

\section{逻辑门电路}

\subsection{逻辑门电路(Logical Gate Circuit)}

基础的逻辑门主要有三种:

\begin{enumerate}
	\item 与门(AND gate)
	      \begin{figure}[H]
		      \centering
		      \begin{circuitikz} \draw
			      (0,2) node[and port] (and) {}
			      (and.in 1) node(x) [anchor=east,xshift=-0.5cm] {$ x $}
			      (and.in 2) node(y) [anchor=east,xshift=-0.5cm] {$ y $}
			      (and.out) node(output) [anchor=west,xshift=0.5cm] {$ x \cdot y $}
			      (x) -- (and.in 1)
			      (y) -- (and.in 2)
			      (and.out) -- (output);
		      \end{circuitikz}
	      \end{figure}

	\item 或门(OR gate):
	      \begin{figure}[H]
		      \centering
		      \begin{circuitikz} \draw
			      (0,2) node[or port] (or) {}
			      (or.in 1) node(x) [anchor=east,xshift=-0.5cm] {$ x $}
			      (or.in 2) node(y) [anchor=east,xshift=-0.5cm] {$ y $}
			      (or.out) node(output) [anchor=west,xshift=0.5cm] {$ x + y $}
			      (x) -- (or.in 1)
			      (y) -- (or.in 2)
			      (or.out) -- (output);
		      \end{circuitikz}
	      \end{figure}

	\item 非门(NOT gate):
	      \begin{figure}[H]
		      \centering
		      \begin{circuitikz} \draw
			      (0,2) node[not port] (not) {}
			      (not.in) node(x) [anchor=east,xshift=-0.5cm] {$ x $}
			      (not.out) node(output) [anchor=west,xshift=0.5cm] {$ \overline{x} $}
			      (x) -- (not.in)
			      (not.out) -- (output);
		      \end{circuitikz}
	      \end{figure}
\end{enumerate}

\begin{tcolorbox}
	\mybox{Exercise}
	设计一个投票表决电路,三个人中有两人赞成即通过。赞成票为1,否决表为0。
	$$
		F(x, y, z) = xy + xz + yz
	$$
	\begin{figure}[H]
		\centering
		\begin{circuitikz} \draw
			(0,0) node[and port] (and3) {}
			(0,2) node[and port] (and2) {}
			(0,4) node[and port] (and1) {}
			(3,2) node[or port, number inputs=3] (or) {}

			(and1.in 1) node(x1) [anchor=east,xshift=-0.5cm] {$ x $}
			(and1.in 2) node(y1) [anchor=east,xshift=-0.5cm] {$ y $}
			(and2.in 1) node(x2) [anchor=east,xshift=-0.5cm] {$ x $}
			(and2.in 2) node(z1) [anchor=east,xshift=-0.5cm] {$ z $}
			(and3.in 1) node(y2) [anchor=east,xshift=-0.5cm] {$ y $}
			(and3.in 2) node(z2) [anchor=east,xshift=-0.5cm] {$ z $}

			(and1.out) node(xy) [anchor=west,xshift=0.3cm] {$ xy $}
			(and2.out) node(xz) [anchor=north,xshift=0.3cm] {$ xz $}
			(and3.out) node(yz) [anchor=west,xshift=0.3cm] {$ yz $}

			(or.out) node(output) [anchor=west,xshift=0.5cm] {$ xy + xz + yz $}

			(x1) -- (and1.in 1)
			(y1) -- (and1.in 2)
			(x2) -- (and2.in 1)
			(z1) -- (and2.in 2)
			(y2) -- (and3.in 1)
			(z2) -- (and3.in 2)
			(and1.out) -- (or.in 1)
			(and2.out) -- (or.in 2)
			(and3.out) -- (or.in 3)
			(or.out) -- (output);
		\end{circuitikz}
	\end{figure}
\end{tcolorbox}

\newpage